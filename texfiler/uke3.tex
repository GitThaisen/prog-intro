
\documentclass{article}
\usepackage[utf8]{inputenc}
\usepackage{enumerate}
\usepackage{enumitem}
\usepackage{fancyvrb}
\usepackage[T1]{fontenc,url}
\usepackage{textcomp, mathpple, esint, listings}
\usepackage{wasysym, graphicx, gensymb}
\usepackage{cases, color} %, tabular}
\usepackage[usenames,dvipsnames]{xcolor}
\usepackage{fancyvrb, pifont}
\setlength\parindent{0pt}
\DefineVerbatimEnvironment{code}{Verbatim}{fontsize=\small}
\DefineVerbatimEnvironment{comment}{Verbatim}{fontsize=\small}
\DefineVerbatimEnvironment{example}{Verbatim}{fontsize=\small}

\newcommand{\tema}[1]{\textbf{Tema: }\textit{#1}\\}
\newcommand{\navn}[1]{\textbf{Navn: }\texttt{#1}\\}

\title{Oppgaver til uke 3 \\ Lister}
\author{NRK}
\date{}


\begin{document}

\maketitle
\noindent

\subsection*{Oppgave 1}
\begin{verbatim}
liste = [0]

liste1 = [1, 2, 3]

liste2 = [4, 4, 5, 4]

liste3 = [“Kristoffer”, “Bjørn Eirik”, “Heidi”, “Per Edvard”, "Einar", “Malin”]

liste4 = [“lister er lure å ha”]

liste5 = []  
\end{verbatim}
Det anbefales å tegne opp \texttt{liste3} på papir for hvert steg den endrer på seg. Deretter burde du sjekke ved å printe den i programmet at du har forstått det riktig.
\begin{enumerate}
    \item Hvor mange elementer har hver liste?
    \item Hvordan kan man printe ut en hel liste? Print ut hele \texttt{liste3}
    \item Hvordan ville du ha printet ut kun \texttt{Heidi} fra \texttt{liste3}?
    \item Hvordan kan du hente ut \texttt{Malin} fra  \texttt{liste3} på to forskjellige måter?
    \item Hvordan kunne du ha lagt til \texttt{Thor Gjermund} på slutten av lista? Og hvordan ville du ha gjort det i starten av lista?
    \item Fjern det første elementet i lista
    \item Erstatt \texttt{Bjørn Eirik} med \texttt{Stian}.
    \item Sorter lista alfabetisk ved å bruke \texttt{sort()}.
    \item Finn ut av hvilken indeks \texttt{Heidi} har i lista.
    \item Sorter lista i reversert alfabetisk med \texttt{sort()}
    \item Print ut lengden til lista
\end{enumerate}

\subsection*{Oppgave 2}
Hva er forskjellen mellom \texttt{sort()} og \texttt{sorted()}? Hvordan vil følgende liste se ut?

\begin{verbatim}
    serier_original = [“skam”, “heimebane”, “exit”]
    print(serier_original.sort())
    print(serier_original)
\end{verbatim}
\noindent
Hvordan vil følgende liste se ut? 
\begin{verbatim}
    serier_original = [“skam”, “heimebane”, “exit”]
    print(serier_original.sorted())
    print(serier_original)
\end{verbatim}


\subsection*{Oppgave 3}

Gitt at vi har følgende lister:
\begin{verbatim}
    serier = [“skam”, “heimebane”, “exit”]
    filmer = ["kongens nei", "the imitation game", "den 12. mann"]
\end{verbatim}

Vi lager så en liste som inneholder begge de to listene.

\begin{verbatim}
    nrktv = [serier, filmer]
\end{verbatim}

\begin{enumerate}
    \item Tegn opp hvordan denne lista ser ut.
    \item Hvordan kan du få tak i \texttt{heimebane}?
    \item Hvordan kan du få tak i \texttt{den 12. mann}?
\end{enumerate}



\end{document}