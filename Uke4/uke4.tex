\documentclass{article}
\usepackage[utf8]{inputenc}
\usepackage{enumerate}
\usepackage{enumitem}
\usepackage{fancyvrb}
\usepackage[T1]{fontenc,url}
\usepackage{textcomp, mathpple, esint, listings}
\usepackage{wasysym, graphicx, gensymb}
\usepackage{cases, color} %, tabular}
\usepackage[usenames,dvipsnames]{xcolor}
\usepackage{fancyvrb, pifont}
\setlength\parindent{0pt}
\DefineVerbatimEnvironment{code}{Verbatim}{fontsize=\small}
\DefineVerbatimEnvironment{comment}{Verbatim}{fontsize=\small}
\DefineVerbatimEnvironment{example}{Verbatim}{fontsize=\small}

\newcommand{\tema}[1]{\textbf{Tema: }\textit{#1}\\}
\newcommand{\navn}[1]{\textbf{Navn: }\texttt{#1}\\}

\title{Oppgaver til uke 4 \\ Lister og for-løkker}
\author{NRK}
\date{}


\begin{document}

\maketitle
\noindent

\subsection*{Oppgave 1}
\begin{enumerate}
    \item Hva er feil her?
    \begin{verbatim}
        for tall in range(10):
        print(f"Tallet er: {tall}")
    \end{verbatim}

    \item Hvor mange vil denne forløkka gå?
    \begin{verbatim}
        for i in range(2, 5):
            print(f"{i}")
    \end{verbatim}
    \item Hvordan kan du bruke en for-loop for å printe ut hver elemenet i en liste som ser slik ut;
    \begin{verbatim}
        serier = ["skam", "heimebane", "exit"]
    \end{verbatim}
\end{enumerate}
\subsection*{Oppgave 2}
\begin{enumerate}
 \item Lag fem heltallsvariabler som inneholer verdiene 0, 1, 2, 3 og 4.
 \item Skriv ut verdiene av variablene.
 \item Lag en tom liste som heter tall.
 \item Legg tallene 0 til 4 inn i listen ved hjelp av en løkke.
 \item Skriv ut variablene i listen ved hjelp av en løkke.
\end{enumerate}


\subsection*{Oppgave 3}
Anta at du har en liste som ser slik ut:

\begin{verbatim}
    liste = [34, 12, 938, 2, 39, 43]
\end{verbatim}

\begin{enumerate}
    \item Finn det elementet som har høyest verdi i lista.
     \item Finn det elementet som har lavest verdi i lista.
      \item Finn summen av alle elementene.
      \item Hvordan kan du bruke slicing for å kun printe ut 12 og 938?
      \item Hvordan kan du printe de tre første tallene?
      \item Hvordan kan du printe de to siste elementene?
      \item Hvordan kan du printe ut fra det andre tallet, til slutten av lista?
\end{enumerate}


\subsection*{Oppgave 4}
Lag en liste med verdiene 1, 2, 3, 4, 5 og 6. Lag en algoritme som reverserer listen. Forsøk å gjøre dette uten å bruke den innebygde funksjonen \texttt{reversed}. Gå deretter gjennom listen med en for-løkke og print ut alle verdiene.

\end{document}

