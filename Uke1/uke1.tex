
\documentclass{article}
\usepackage[utf8]{inputenc}
\usepackage{enumerate}
\usepackage{enumitem}
\usepackage{fancyvrb}
\usepackage[T1]{fontenc,url}
\usepackage{textcomp, mathpple, esint, listings}
\usepackage{wasysym, graphicx, gensymb}
\usepackage{cases, color} %, tabular}
\usepackage[usenames,dvipsnames]{xcolor}
\usepackage{fancyvrb, pifont}

\DefineVerbatimEnvironment{code}{Verbatim}{fontsize=\small}
\DefineVerbatimEnvironment{comment}{Verbatim}{fontsize=\small}
\DefineVerbatimEnvironment{example}{Verbatim}{fontsize=\small}

\newcommand{\tema}[1]{\textbf{Tema: }\textit{#1}\\}
\newcommand{\navn}[1]{\textbf{Navn: }\texttt{#1}\\}

\title{Oppgaver til uke 1 \\ Installering og print}
\date{}
\author{NRK}

\begin{document}

\maketitle
\noindent

\subsection*{Installasjon}
\begin{itemize}
    \item Installere Python 3.7 fra https://www.python.org/downloads/.
    \item Installere Visual Studio Code fra https://code.visualstudio.com/.
    \item Installere Python extension i VSCode.
\end{itemize}


\subsection*{Oppgave 1}
\begin{enumerate}
    \item Lag en fil i VSCode som du kaller for \textit{hallo.py}
    \item Skriv \texttt{print("Hallo python!")} i filen.
    \item Lagre
    \item Kjør programmet. Sjekk at det printes ut.
\end{enumerate}


\subsection*{Oppgave 2}
Hva blir printet ut? Hvorfor blir det printet ut?

\begin{verbatim}
print("5 + 6")

print(5 + 6)
\end{verbatim}


\subsection*{Oppgave 3}
Finn dokumentasjonen til print på python.org 

\subsection*{Oppgave 4}
\begin{enumerate}
    \item Print ut ditt navn, tlf og adresse på samme linje.
    \item Print ut ditt navn, tlf og adresse på hver sin linje.
\end{enumerate}
 \noindent
\textit{Ekstra: Spesialkaraktereren \textbackslash n betyr linjeskift. Se om du kan bruke denne for å printe ut på hver sin linje i en print-setning.}




\end{document}
