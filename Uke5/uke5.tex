
\documentclass{article}
\usepackage[utf8]{inputenc}
\usepackage{enumerate}
\usepackage{enumitem}
\usepackage{fancyvrb}
\usepackage[T1]{fontenc,url}
\usepackage{textcomp, mathpple, esint, listings}
\usepackage{wasysym, graphicx, gensymb}
\usepackage{cases, color} %, tabular}
\usepackage[usenames,dvipsnames]{xcolor}
\usepackage{fancyvrb, pifont}
\setlength\parindent{0pt}
\DefineVerbatimEnvironment{code}{Verbatim}{fontsize=\small}
\DefineVerbatimEnvironment{comment}{Verbatim}{fontsize=\small}
\DefineVerbatimEnvironment{example}{Verbatim}{fontsize=\small}

\newcommand{\tema}[1]{\textbf{Tema: }\textit{#1}\\}
\newcommand{\navn}[1]{\textbf{Navn: }\texttt{#1}\\}

\title{Oppgaver til uke 5 \\ If-setninger}
\author{NRK}
\date{}


\begin{document}

\maketitle
\noindent

\subsection*{Oppgave 1}
Anta at du har en variabel \texttt{alder}.

Lag en if-setning som printer ut følgende dersom alder er lik eller større enn 18;
\begin{verbatim}
Du er myndig
\end{verbatim}

og printer ut følgende dersom alder ikke er det;
\begin{verbatim}
Du er ikke myndig
\end{verbatim}

Test programmet ved å sette verdien til alder til 13, 18 og 23 og sjekk at den gir ut riktig resultat.

\subsection*{Oppgave 2}

\begin{verbatim}
tall1 = 4
tall2 = 4

if tall1 > tall2:
 print(f"{tall1} er storre enn {tall2}")
else:
 print(f"{tall2} er storre enn {tall1}")
 \end{verbatim}
 
 \begin{enumerate}
 \item Hva vil skrives ut?
 \item Utvid og endre på koden slik at den vil skrive ut for like tall.
 \end{enumerate}
 


\subsection*{Oppgave 3}
Hva evalueres de ulike uttrykkene til?
\begin{enumerate}
\item 
\begin{verbatim}
True
False
True and True
True and False
True or True
True or False	
False or False
not True 
not False
True and not True
True and not False
not True or not False
\end{verbatim}

\item
\begin{verbatim}
3>3 
3>=3
3<3 
3<=3
3==3
3!=3
\end{verbatim}
\end{enumerate}

\subsection*{Oppgave 4}
Du skal skrive ut pris for bussbillett til reisende. Dersom du er under 16 \textit{eller} over 65 får du halv pris. Du skal bruke følgende variabel:

\begin{verbatim}
standard_pris = 50
\end{verbatim}

Lag en variabel \texttt{alder} og bruk denne for å print ut hvor mye vedkommende skal betale.

\subsection*{Oppgave 5}
Anta at du har en liste med følgende verdier:

\begin{verbatim}
liste = [34, 2, 24, 4, 23, 45, 5, 7, 35, 3]
\end{verbatim}
I tillegg så har du en maksverdi
\begin{verbatim}
maksverdi = 10
\end{verbatim}

Ved å bruke en for-løkke og en if-setning, print ut alle tall som er mindre enn \texttt{maksverdi}.

Eksempel på print med \texttt{liste} og \texttt{maksverdi} med gitte verdier:
\begin{verbatim}
2
4
5
7
3
\end{verbatim}

\subsection*{Oppgave 6}
Anta at du har en liste med følgende verdier:

\begin{verbatim}
liste = [-34, 2, -24, 4, 23, -45, 5, 7, 35, 3]
\end{verbatim}

\begin{enumerate}
\item Bruk en for-løkke og en if-setning for å finne den minste verdien og skriv den ut. Du skal ikke bruke Python sin innebygde \texttt{min}-funksjon.
\item Gjør det samme for den største verdien.
\item Summer alle tallene i listen og print ut.
\end{enumerate}

\subsection*{Oppgave 1}

\end{document}